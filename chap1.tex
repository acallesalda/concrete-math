\documentclass[a4paper]{article}
\usepackage{amsmath}
\usepackage{amssymb}
\usepackage{bbm}
\usepackage{geometry}
%\geometry{left=2.5cm,right=2.5cm,top=2.5cm,bottom=2.5cm}

\DeclareMathOperator{\E}{\mathbb{E}}
\newcommand{\Var}{\mathrm{Var}}
\newcommand{\normal}{\mathcal{N}}

\begin{document}

\title{Chapter 1: Recurrent Problems}
\author{Alejandro Calle-Saldarriaga\\
	\ttfamily{callesaldarr@wisc.edu}}

\section*{Exercise 2.}

Try visualizing for $1, 2, 3$ pegs, and you will get that 
$T_0 = 0, T_1 = 2, T_2 = 8$, and that the winning strategy 
is to transfer $n-1$ pegs to peg $B$ via peg $C$, which takes 
$T_{n-1}$ moves. Now, transfer the biggest peg to $C$, which takes
one move, and then trasnfer the $n-1$ pegs in $B$ to $A$, which takes 
$T_{n-1}$ moves. Transfer biggest from $C$ to $B$, taking one move, and 
transfer the other pegs to $B$, taking again $T_{n-1}$ moves, that is

\begin{align*}
    T_0 & = 0 \\
    T_{n} & = 3T_{n-1} + 2
\end{align*}

To solve this recurrence, we sum $1$ to each side (the idea here 
is that we want to factor $T_{n+1}$, and summing $1$ will make everything a 
factor of $3$), so we have 

\begin{align*}
    T_0 + 1 & = 1 \\
    T_{n} + 1 & = 3T_{n-1} + 3
\end{align*}

and letting $U_n = T_{n} + 1$, we get 

\begin{align*}
    U_0 & = 1 \\
    U_n & = 3U_{n-1}
\end{align*}

so clearly $U_n = 3^n$ and $T_n = 3^n - 1$. To prove that this is indeed the 
case, let's now use inducction. For $n=0$, $T_0 = 3^0 - 1 = 0$, and assume 
$T_{n-1} = 3^{n-1} - 1$, so 

\begin{equation*}
    T_n = 3T_{n-1} + 2 = 3(3^{n-1}-1) + 2 = 3^n - 3 + 2 = 3^n - 1
\end{equation*}


\end{document}